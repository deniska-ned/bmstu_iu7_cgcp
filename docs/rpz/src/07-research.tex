\chapter{Экспериментальный раздел}

    В данном разделе описан эксперимент по сравнению временных характеристик последовательной и параллельной реализации алгоритма.
    
    \section{Цель эксперимента}
    
        Целью эксперимента является оценка временной эффективности параллельной реализации алгоритма обратной трассировки лучей.
    
    \section{Технические характеристики}

        Технические характеристики устройства, на котором выполнялось исследование:

        \begin{itemize}
        	\item процессор: Intel Core™ i5-8250U \cite{i5} CPU @ 1.60GHz;
        	\item память: 8 GiB;
        	\item операционная система: Fedora \cite{fedora} Linux \cite{linux} 21.1.4 64-bit.
        \end{itemize}
        
        Исследование проводилось на ноутбуке, включенном в сеть электропитания. Во время тестирования ноутбук был нагружен только встроенными приложениями окружения рабочего стола, окружением рабочего стола, а также непосредственно системой тестирования.
        
    \section{Описание эксперимента}

        Была реализована функция параллельного синтеза сцены. Для этого была использована библиотека \texttt{OpenMP} \cite{omp}, директива препроцессора \texttt{\#pragma omp parallel for}. Данная директива препроцессора преобразует код для выполнения итераций цикла параллельно. 

        В рамках данного эксперимента произведена оценка влияния размера изображения и количества объектов сцены на время работы программы. Для сравнения были синтезированы квадратные изображения с размерами равными [100, 200, 500, 1000, 2000]. Количество объектов сцены задавалось равным [2, 4, 8, 16] штук.
 
    \section{Результат эксперимента}

        В таблице \ref{tbl:time} приведены экспериментально полученные значения временных характеристик работы алгоритма в зависимости от размера синтезируемого изображения и количества объектов сцены.
        
\begin{table}[ht]
	\small
	\begin{center}
		\caption{Замеры времени для изображений с различными размерностями}
		\label{tbl:time}
		\begin{tabular}{|c|c|c|c|}
        \hline
        & & \multicolumn{2}{c|}{Время (мс)} \\
        \cline{3-4}
        \raisebox{1.5ex}{Кол-во объектов} & \raisebox{1.5ex}{Размер сцены} & Послед. реал. & Паралл. реал. \\
        \hline
        & \texttt{100x100} & 49 & 21 \\
        \cline{2-4}
        & \texttt{200x200} & 301 & 83  \\
        \cline{2-4}
        \texttt{2} & \texttt{500x500} & 1606 & 442 \\
        \cline{2-4}
        & \texttt{1000x1000} & 5068 & 1853 \\
        \cline{2-4}
        & \texttt{2000x2000} & 20428 & 8349 \\
        \hline
        & \texttt{100x100} & 135 & 49  \\
        \cline{2-4}
        & \texttt{200x200} & 380 & 168 \\
        \cline{2-4}
        \texttt{4} & \texttt{500x500} & 2024 & 672 \\
        \cline{2-4}
        & \texttt{1000x1000} & 7840 & 2672 \\
        \cline{2-4}
        & \texttt{2000x2000} & 32666 & 10859 \\
        \hline
        & \texttt{100x100} & 135 & 38 \\
        \cline{2-4}
        & \texttt{200x200} & 544 & 165 \\
        \cline{2-4}
        \texttt{8} & \texttt{500x500} & 3341 & 1072 \\
        \cline{2-4}
        & \texttt{1000x1000} & 13009 & 4233 \\
        \cline{2-4}
        & \texttt{2000x2000} & 51708 & 17864 \\
        \hline
        & \texttt{100x100} & 272 & 74 \\
        \cline{2-4}
        & \texttt{200x200} & 1098 & 302 \\
        \cline{2-4}
        \texttt{16} & \texttt{500x500} & 6693 & 1864 \\
        \cline{2-4}
        & \texttt{1000x1000} & 26108 & 7451 \\
        \cline{2-4}
        & \texttt{2000x2000} & 102328 & 31217 \\
        \hline
        \end{tabular}
	\end{center}
\end{table}

\begin{figure}[!h]
    \centering
    \begin{tikzpicture}
        \begin{axis}[
            axis lines=left,
            xlabel={Размер сцены, \( \sqrt{n} \)},
            ylabel={Время, мс},
            legend pos=north west,
            ymajorgrids=true
        ]
            \addplot[color=red] coordinates {(100,7)(200,17)(500,40)(1000,71)(2000,142)};
            \addplot[color=green] coordinates {(100,11)(200,19)(500,44)(1000,88)(2000,180)};
            \addplot[color=blue] coordinates {(100,11)(200,23)(500,57)(1000,114)(2000,227)};
            \addplot[color=black] coordinates {(100,16)(200,33)(500,81)(1000,161)(2000,320)};
            \legend{Кол-во объектов: 2, Кол-во объектов: 4, Кол-во объектов: 8, Кол-во объектов: 16}
        \end{axis}
    \end{tikzpicture}
    \captionsetup{justification=centering}
    \caption{Зависимость времени работы алгоритма от размера сцены}
    \label{plt:time_by_size}
\end{figure}
        
\begin{figure}[!h]
    \centering
    \begin{tikzpicture}
        \begin{axis}[
            axis lines=left,
            xlabel={Количество объектов сцены, \( n \)},
            ylabel={Время, мс},
            legend pos=north west,
            ymajorgrids=true
        ]
            \addplot[color=red] coordinates {(2,49)(4,135)(8,135)(16,272)};
            \addplot[color=green] coordinates {(2,301)(4,280)(8,544)(16,1098)};
            \addplot[color=blue] coordinates {(2,1606)(4,2024)(8,3341)(16,6693)};
            \addplot[color=purple] coordinates {(2,5068)(4,7840)(8,13009)(16,26108)};
            \addplot[color=black] coordinates {(2,20428)(4,32666)(8,51708)(16,102328)};
            \legend{Размер 100x100, Размер 200x200, Размер 500x500, Размер 1000x1000, Размер 2000x2000}
        \end{axis}
    \end{tikzpicture}
    \captionsetup{justification=centering}
    \caption{Зависимость времени работы алгоритма от количества объектов сцены}
    \label{plt:time_by_count}
\end{figure}

        По данным, приведенным в таблице \ref{tbl:time} построены графики зависимостей времени работы алгоритма от размера изображения (рисунок \ref{plt:time_by_size}) и от количества объектов сцены (рисунок \ref{plt:time_by_count}). Зависимость времени работы алгоритма от размера сцены является квадратичной, то есть зависит как $O(n^2)$, где n - ширина (высота) квадратного изображения. Время синтеза изображения примерно прямо пропорционально количеству объектов сцены, то есть зависит как $O(n)$, где $n$ - количество объектов сцены.


        Стоит отметить, что параллельная реализация алгоритма оказалась более эффективной. Её время работы в среднем в 3 раза меньше, чем у последовательного алгоритма.
        
\clearpage

    \section{Вывод}

        В данном разделе было произведено экспериментально сравнение временных характеристик реализованного программного обеспечения.
        
        Время работы алгоритма имеет примерно квадратичную зависимость от размера синтезируемого изображения и линейную зависимость от количества объектов сцены.
        
        Наиболее эффективной по времени оказалась многопоточная реализация алгоритма. В среднем время ее работы меньше 3 раза, чем последовательной реализации.