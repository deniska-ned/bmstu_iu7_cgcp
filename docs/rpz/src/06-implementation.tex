\chapter{Технологический раздел}

    В данном разделе будут представлены средства разработки программного обеспечения, детали реализации и процесс сборки разрабатываемого программного обеспечения.
    
    \section{Выбор средств программной реализации}

        В качестве языка программирования для разработки программного обеспечения был выбран язык \texttt{C++} \cite{cpp}. Данный выбор обусловлен тем, что данный язык предоставляет весь функционал требуемый для решения поставленной задачи.

        Для создания пользовательского интерфейса ПО был использован фреймворк \texttt{QT} \cite{qt}. Данный фреймворк содержит в себе объекты, позволяющие напрямую работать с пикселями изображения, а  так же возможности создания интерактивных пользовательских интерфейсов, что позволит в интерактивном режиме управлять изображением.

        В качестве стиля кода был выбран стиль Mozilla \cite{fmtmozilla}. Для проведения автоформатирования был выбран инструмент \texttt{clang-format} \cite{clangfmt}, так как он поддерживает работу в командной строке, а так же реализован в качестве плагина для популярных ide.
        
        Для отслеживания утечек памяти был выбран инструмент \texttt{Valgrind} \cite{valgrind}.
        
        Для сборки программного обеспечения использовался инструмент \texttt{CMake} \cite{cmake}.
        
        В качестве среды разработки был выбран текстовый редактор \texttt{vim} \cite{vim}, поддерживающий работу в командной строке, а так же установкy плагинов \cite{vimawesome}, в том числе для работы с \texttt{C++} и \texttt{CMake}.

    \section{Процесс сборки приложения}

        Для сборки программного обеспечения использовался инструменты \texttt{CMake} \cite{cmake}.

        Для сборки приложения необходимо в командной строке, находясь в директории проекта, выполнить следующие команды.
        
        \listingfile{build.sh}{}{Сборка реализуемого программного обеспечения}{}

    \section{Пользовательский интерфейс}
    
        Интерфейс реализуемого ПО представлен на рисунках \ref{img:ui} – \ref{img:ui_new_obj_02}.

        \imgw{ui}{ht!}{\textwidth}{Интерфейс программы. Общий план}
        
        На рисунке \ref{img:ui} изображен общий вид программы. Слева находится экран для вывода изображения. Справа -- интерфейс настройки параметров изображения. Рассмотрим каждую из вкладок настройки отдельно.
        
        \imgs{ui_render}{ht!}{3}{Интерфейс программы. Вкладка настроек рендера}
        
        На рисунке \ref{img:ui_render} представлена вкладка генерирования изображения. На ней доступен выбор качества генерируемого изображения и кнопка запуска генерирования изображения.
        
        \imgs{ui_constructor}{ht!}{3}{Интерфейс программы. Вкладка настроек сцены}
        
        На рисунке \ref{img:ui_constructor} представлена вкладка настройки параметров сцены. Она включает в себя задание цвета заднего фона, настройку положения камеры, а также содержит список объектов сцены с возможностью их удаления.
        
        \imgs{ui_new_obj_01}{ht!}{3}{Интерфейс программы. Вкладка добавления нового объекта. Часть 1}
        
        \imgs{ui_new_obj_02}{ht!}{3}{Интерфейс программы. Вкладка добавления нового объекта. Часть 2}
        
        На рисунках \ref{img:ui_new_obj_01} -- \ref{img:ui_new_obj_02} представлена вкладка добавления новых объектов для сцены. Она включает в себя выбор фигуры, ее материала и текстуры. Для прозрачного материала дана возможность ввода коэффициентов формула Зельмейера.
    
    \clearpage
        
    \section{Примеры работы приложения}
    
        На рисунках \ref{img:exp_01} -- \ref{img:exp_03} представлены примеры работы приложения.
        
        На рисунке \ref{img:exp_01} представлена сцена с видимой дисперсией. В сцене используется стеклянный шар на фоне сферы с шахматным узором. Для наблюдения дисперсия камера расположена близко с стеклянной сфере.
        
        \imgw{exp_01}{ht!}{\textwidth / 2}{Пример работы приложения. Сцена с дисперсией}
        
        На рисунке \ref{img:exp_02} представлена сцена с металлическим и матовым шаром. В металлическом фаре наблюдается отражаются другие объекты сцены.
        
        \imgw{exp_02}{ht!}{\textwidth / 2}{Пример работы приложения. Сцена с металлическим и матовым шаром}
        
        На рисунке \ref{img:exp_03} представлена сцена с двумя источниками света: желтого и синего.
        
        \imgw{exp_03}{ht!}{\textwidth / 2}{Пример работы приложения. Сцена с источниками света}
    
    \clearpage
    
    \section{Вывод}

        В данном разделе были представлены средства разработки программного обеспечения, детали реализации, пользовательский интерфейс и процесс сборки разрабатываемого программного обеспечения.