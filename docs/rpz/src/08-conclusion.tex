\chapter*{ЗАКЛЮЧЕНИЕ}
\addcontentsline{toc}{chapter}{ЗАКЛЮЧЕНИЕ}

    В ходе курсового проекта было разработано программное обеспечение, предоставляющее возможность визуализации дисперсии света на прозрачных предметах. Разработанное программное обеспечение предоставляет функционал для изменения интервалов построения поверхностей, задания цвета и свойств материала поверхности, а так же задания и изменения в процессе работы положения точки наблюдения и источников света по их характеристикам (положению, интенсивности) в интерактивном режиме. В процессе выполнения данной работы были выполнены следующие задачи:


    \begin{itemize}
        \item изучение явления дисперсии с физической точки зрения;
        \item определение зависимостей влияющей на преломление лучей света при прохождении через прозрачную поверхность;
        \item описание существующих алгоритмов построения реалистичных изображений;
        \item выбор реализуемого алгоритмы;
        \item приведение схемы реализуемых алгоритмов;
        \item проектирование архитектуры и графического интерфейса программы;
        \item реализация структур данных и алгоритмов;
	    \item описание структуры разрабатываемого ПО;
        \item исследование производительности программы.
    \end{itemize}