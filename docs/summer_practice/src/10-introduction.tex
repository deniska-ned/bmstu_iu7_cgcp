\section*{Введение}
\addcontentsline{toc}{chapter}{Введение}

Дисперсия - это chapterе разбиения белого пучка света на его цветные
составляющие при прохождении через прозрачные поверхности. Наиболее известным
экспериментом, показывающим это явление, является пропускание белого пучка
света через призму и наблюдение светового спектра на экране. Явление дисперсии
также иллюстрируют радуга и блеск драгоценных камней.

Явление дисперсии занимало ума людей столетиями. Инженеры, проектирующие
оптические приборы, стремились минимизировать ее проявление в своих приборах.
В то время как ювелиры находились в постоянном стремлении преумножить блеск
своих драгоценных камней.

Дисперсия повсюду встречается в нашей жизни, это явление полно изучено
с физической точки зрения. Поэтому при построении изображений, претендующих
на фотореалистичность, нельзя обойти стороной учтение этого явления.

Целью моего проекта является разработка программного обеспечения
для визуализации трехмерных объектов и наблюдения дисперсии света
на прозрачных поверхностях.

Для достижения поставленной цели необходимо решить следующие задачи:

\begin{itemize}
    \item изучение явления дисперсии с физической точки зрения;
    \item определение зависимостей влияющей на преломление лучей света
    при прохождении через прозрачную поверхность;
    \item анализ и выбор алгоритмов построения реалистичного трехмерного
    изображения;
    \item проектирование архитектуры и графического интерфейса программы;
    \item реализация структур данных и алгоритмов;
    \item исследование производительности программы.
\end{itemize}
